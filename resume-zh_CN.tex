% !TEX TS-program = xelatex
% !TEX encoding = UTF-8 Unicode
% !Mode:: "TeX:UTF-8"

\documentclass{resume}
\usepackage{zh_CN-Adobefonts_external} % Simplified Chinese Support using external fonts (./fonts/zh_CN-Adobe/)
%\usepackage{zh_CN-Adobefonts_internal} % Simplified Chinese Support using system fonts
\usepackage{linespacing_fix} % disable extra space before next section
\usepackage{cite}

\begin{document}
\pagenumbering{gobble} % suppress displaying page number

\name{刘天伟}

% {E-mail}{mobilephone}{homepage}
% be careful of _ in emaill address
\contactInfo{186-1132-0580}{liutianweidlut@qq.com}{10年工作经验 | github.com/tianweidut}{1987.8}

\section{教育经历}
\datedsubsection{\textbf{大连理工大学} \quad 软件工程(本科),创新院-计算机应用技术(硕士)}{2006.9 - 2013.6}

\section{工作经历}
\datedsubsection{\textbf{星鲸} \quad 软件工程师}{2022.3-今}
\begin{itemize}
    \item 星鲸是一家专注于开源MLOps/LLMOps平台的天使轮创业公司。
    \item 首位全职工程师,负责构建Starwhale。它是以模型、数据集和运行时为基础,实现了运行在Standalone/Server/Cloud三种实例模式下,能够满足Model Evaluation/Live Demo/Fine-tuning等ML特定领域功能需求的开源MLOps/LLMOps平台(github.com/star-whale/starwhale)。
\end{itemize}
\datedsubsection{\textbf{旷视}  \quad 主任架构师(T5.1)}{2017.9-2022.2}
\begin{itemize}
	\item 同时担任中台-平台开发部、算法产品部-基础平台两个团队的Leader,团队规模50人。
\end{itemize}
\begin{itemize}
  \item \textbf{私有云平台DevOps}:自研容器编排、集群自动化运维、自定义交付包等核心功能,在ToB领域、私有化、重交付场景中实现标准化交付、自动化运维和最佳实践开箱即用。4年时间里,完成多个BG、10+产品线、1000+个国内外项目的应用托管和落地实施,是旷视所有私有化项目的基础要素。
  \item \textbf{云原生PaaS平台MCD}:面向全体研发,在混合云/私有云/边缘计算场景下,基于k8s构建的application概念,提供CI和周边基础设施,形成内部完善的、面向云原生的、GitOps驱动的PaaS平台。上线两年,实现1300+个应用的托管,是旷视跨BG使用最多的一个基础技术平台。
  \item \textbf{AI生产力平台Brain++}:基于k8s,提供workspace/rlanuch/rrun/rjob等训练方式的抽象、100PB的底层OSS存储和相关基础设施。集群规模4000台机器、13000张显卡,托管旷视所有模型训练任务。
  \item \textbf{私有化交付工具链MDT}:以私有云DevOps平台为核心,构建包括BootOS(自定义OS发行版)、SDC(软件包托管)、DevOps-Swarm(多集群管理)、DevOps-Import(运维巡检)、编排托管、ATP(自动化测试)等工具,实现旷视完整的面向ToB/ToG的私有化项目交付工具链,实现高效、标准化交付。
  \item \textbf{旷视加密授权体系}:基于CodeMeter和Sentinel LDK商业产品,构建统一、安全、全场景的加密授权方案与平台,承载旷视内部所有加密授权需求。
\end{itemize}
\begin{itemize}
  \item 旷视CBG、中台技术部、算法产品部技术委员会成员,负责晋升评审、项目规划、立项评审等工作。
	\item 从0-1搭建平台组,4年多经历从CoreTeam IC、CoreTeam DevOps组leader、CBG平台开发部leader到公司级中台-平台开发部leader的变化,实现平台型工具在旷视内部影响力的最大化。2021.1接管Brain++团队,完成年度技术规划、目标梳理和团队重构,并协助其获得旷视2021年度优秀团队。
	\item 在旷视4次年度考评中3次A绩效(A为最高档),2020.9由T4.2跨级晋升到T5.1。
\end{itemize}

\datedsubsection{\textbf{豆瓣} \quad 高级系统软件工程师}{2013.7-2017.9}
\begin{itemize}
  \item 平台组,DAE核心开发。DAE是豆瓣内部的PaaS平台,托管豆瓣所有的服务。
  \item 负责DAE中在线和离线任务AutoScale调度、集群扩缩容机制、Cron/Daemon/MQ/Admin等离线任务的设计与实现,主导DAE容器化改造和DAE Python Framework的开发。
  \item 全站HTTP 502报警首要响应人,处理大量线上故障;负责豆瓣第三方网站爬虫服务。
\end{itemize}

\datedsubsection{\textbf{微软} \quad 实习生}{2012.7-2012.10}
\begin{itemize}
  \item BingDesktop组,负责定制浏览器的调研与实现,包括类VIM快捷键、网页阅读模式、预加载机制等。
\end{itemize}

\datedsubsection{\textbf{Google Summer of Code} \quad 实习生}{2012.3-2013.4}
\begin{itemize}
  \item UMIT组。Remote方式与多国同事协同开发的针对网络审查监控的开源项目OpenMonitor,负责Desktop Agent部分,包括搭建P2P网络、设计Dashboard/SoftwareUpdate/BugReport等。
\end{itemize}

\section{分享|技能}
\begin{itemize}
    \item 2016 \quad \textbf{Docker在豆瓣的实践 } - 灵雀云Docker Meetup
    \item 2017 \quad \textbf{容器化技术在豆瓣微服务化中的实践 } - Women Who Code走进豆瓣活动
    \item 2020 \quad \textbf{旷视城市大脑的私有云平台建设及稳定性保障} - GOPS全球运维大会
    \item 2021 \quad \textbf{旷视AI产品背后的研发效能体系建设} - EE卓越工程生产力大会
\end{itemize}
\begin{itemize}
    \item 丰富的在线debug和解决问题经历,工作语言Python/Golang/JS,积极关注Rust。
    \item 对Pass/DevOps/MLOps/AI Infra/云原生等保持一贯的热爱,良好的平台型工具产品思维。
    \item 大量的在公司内部推广、布道基础平台经历,对多团队合作、平台型组织运作有着深入的认知。
    \item 目标绩效、梯队建设等管理工具熟练掌握,崇尚教练式技术,5年中小技术团队管理经验(3-50人)。
\end{itemize}

\end{document}